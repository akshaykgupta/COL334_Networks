\documentclass[12pt]{article}
\usepackage[margin=1.2in]{geometry}
\usepackage{amsmath}
\usepackage{graphicx}
\usepackage{caption}
\usepackage{subcaption}
\begin{document}
\title{COL334 - Assignment 1\\ Internet Architecture}
\author{Akshay Kumar Gupta\\\texttt{2013CS50275} \and  Barun Patra\\\texttt{2013CS10773} \and Haroun Habeeb\\\texttt{2013CS10225}}
\date{}
\maketitle
\noindent
{\bfseries Q3a.} The frequency distribution for the number of hops from different traceroute servers to the given destinations is plotted in `Q3.xlsx' in the worksheet `number-of-hops'. From the chart we can observe that:
\begin{itemize}
\item The number of hops between nodes in the same continent is lower that between nodes in different continents.
\item The number of hops required to reach Google (Mountain View) is lower than the number of hops required to reach Facebook (Menlo Park).
\end{itemize}
~ \\
{\bfseries Q3b.} The frequency distribution for the latency is plotted in `Q3.xlsx' in the worksheet `latency' and correlation between number of hops and latency is recorded in the worksheet `correlation'. The correlation is positive, which means that latency increases with the number of hops.
\\ \\ \\
{\bfseries Q3c.} 
\\ \\ \\
{\bfseries Q3d.} The number of hops and latency incurred when running traceroute from the local ISP are recorded in `Q3.xlsx' in the worksheet `local-isp'. There does not seem to be any bottleneck, so no conclusion can be drawn about the greatest source of latency.
\\ \\ \\
{\bfseries Q3e.} The routes to each destination server have roughly the same number of hops. This means that the local ISP is well connected to different parts of the world.
\end{document}